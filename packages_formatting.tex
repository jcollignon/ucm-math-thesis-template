%\usepackage[utf8]{inputenc}

%%%%%%%%%%%%%%%%%%%%%%%%%%%%%
% Packages required for page formatting

\usepackage[letterpaper, top=1in, bottom=1.25in, left=1.5in, right=1in, marginparsep=0in, footskip=0.75in, headheight=15pt]{geometry} % Defines the global margins of the document
% Main text body:
% Top and right margins are 1in.
% Bottom margin is 1.25in.
% Left margin is 1.5in.
% Space between text and right margin is 0 (marginparsep=0in)
% Top of the header is 0.5in from the top of the page by default
% Bottom of the footer is 0.5in from the bottom of the page.  Configure this by setting footskip to be 0.5in less than the bottom margin.
% Header height just defines the size of the header box.  It's set to 15pt to suppress a warning and doesn't actually cause anything to change.
\usepackage{fancyhdr} % necessary to change heading styles
\usepackage{caption} % Loads a function that will help shrink your list of figures and tables

%%%%%%%%%%%%%%%%%%%%%%%%%%%%%
% Font Packages
% Comment out unwanted fonts
% Default is the Computer Modern font.

%\usepackage[T1]{fontenc}
%\usepackage{mathpazo}
%\usepackage{utopia}
%\renewcommand*\familydefault{\sfdefault} 

%%%%%%%%%%%%%%%%%%%%%%%%%%%%%
% Recommended packages
% Modify as needed

\usepackage{amsmath,amsthm,amssymb,amsfonts, color, comment, graphicx, environ}
\usepackage{epsfig}
\usepackage{epstopdf}
\usepackage{enumitem}
\usepackage{changepage}
\usepackage{appendix}
\usepackage[hidelinks]{hyperref} % required to keep text uncolored but still hyperlinkable
\usepackage{mathrsfs}
\usepackage{bbm}
\usepackage{booktabs}
\usepackage{url}
\usepackage{ulem}
\usepackage{setspace}
\usepackage{multirow}
\usepackage{multicol}
\usepackage{indentfirst}
\usepackage{wrapfig}
\usepackage{float}
\usepackage{comment} 
\usepackage{subcaption}
\usepackage{longtable}

\usepackage{etoolbox}

\usepackage{pdfpages}
\usepackage{xcolor}
\usepackage{makecell}

% Use a referencing package
%\usepackage{biblatex}
\usepackage{cite}

% Other packages
%\usepackage{enumerate}
%\usepackage{breqn}
%\usepackage{gensymb}

%%%%%%%%%%%%%%%%%%%%%%%%%%%%%%
% Chanigng styles for pages
% Necessary for ensuring page numbers are in the right places

% Positioning of text for Abstract
% \fancypagestyle{abstract}{%
% \fancyhf{} % start with empty header and footer
% \fancyfoot[C]{\thepage} % Place the page number on the bottom center
% %\renewcommand{\headrulewidth}{0pt}% 
% \renewcommand{\footrulewidth}{0in}%
% }

% Page number position for all other Arabic numbered pages.
% This is user-defined:
\fancypagestyle{mainhead}{%
\fancyhf{} % start with empty header and footer
\fancyhead[R]{\thepage} % Place the page number on the top right
\renewcommand{\headrulewidth}{0pt} % remove header line
\renewcommand{\footrulewidth}{0pt}% remove footer line
%\addtolength{\topmargin}{0in}
%\setlength{\headheight}{0in}
}

% Page number for beginning pages of Chapter, Appendix, and Reference sections:
% The "plain" style is the default, but you can define your own style using the code template below:
% \fancypagestyle{chapterhead}{%
% \fancyhf{} % start with empty header and footer
% \fancyfoot[C]{\thepage} % Place the page number on the bottom center
% %\renewcommand{\headrulewidth}{0pt}% 
% \renewcommand{\footrulewidth}{0in}%
% }


% Code for dealing with referencing the References section in the table of contents
\patchcmd{\thebibliography}{\section*}{\section}{}{}
\renewcommand\bibname{References}


%%%%%%%%%%%%%%%%%%%%%%%%%%%%%%
% Define your own functions

% Definitions for fancy math symbols
\def\ones{\mathbbm{1}}
\def\Re{\mathbb{R}}
\def\Ze{\mathbb{Z}}




